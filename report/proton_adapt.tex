\usepackage{eskdchngsheet}
%\usepackage{python}
\usepackage{multirow}
\usepackage{amsmath}
\usepackage{cite}
% \usepackage{tocloft}
\usepackage{amsfonts}
\usepackage{amssymb}
\usepackage{pdfpages}
\usepackage{booktabs} % For \toprule, \midrule and \bottomrule
\usepackage{siunitx} % Formats the units and values
\usepackage{pgfplotstable} % Generates table from .csv
\usepackage{cmap}
\usepackage{color} \definecolor{darkgreen}{rgb}{0,.5,0}
%\usepackage[unicode,colorlinks,filecolor=blue,citecolor=darkgreen]{hyperref}
\usepackage{lscape}
\usepackage{eskdfreesize}
%\usepackage[T2A]{fontenc}
%\usepackage{pscyr}
\sisetup{
  round-mode          = places, % Rounds numbers
  round-precision     = 2, % to 2 places
}

\renewcommand{\ESKDagreedName}{%
  \cyr\CYRS\CYRO\CYRG\CYRL\CYRA\CYRS\CYRO\CYRV\CYRA\CYRN\CYRO} % "СОГЛАСОВАНО" заглавными
\renewcommand{\ESKDapprovingName}{%
  \cyr\CYRU\CYRT\CYRV\CYRE\CYRR\CYRZH\CYRD\CYRA\CYRYU} % "УТВЕРЖДАЮ" заглавными
\renewcommand{\ESKDapprovedName}{%
   \cyr\CYRU\CYRT\CYRV\CYRE\CYRR\CYRZH\CYRD\CYRE\CYRN} % "УТВЕРЖДЕН" заглавными
\renewcommand{\ESKDapprovingSheetName}{%
  \cyr\CYRL\cyri\cyrs\cyrt\cyru\cyrt\cyrv\cyre\cyrr\cyrzh\cyrd\cyre\cyrn\cyri\cyrya}

%\ESKDcompany{ПАО <<ПРОТОН-ПМ>>}
%\ESKDgroup{ПАО <<ПРОТОН-ПМ>>}
%\ESKDcompany{ООО НТЦ <<ТУРБОПНЕВМАТИК>>}
%\ESKDgroup{ООО НТЦ <<ТУРБОПНЕВМАТИК>>}
%\ESKDtitleApprovedBy{Главный инженер}{Т. Н. Компанец}
%\ESKDtitleApprovedBy{Технический директор}{А. А. Снитко}

\ESKDauthor{Целищев}
\ESKDchecker{}
\ESKDnormContr{}
\ESKDapprovedBy{}

%\ESKDtitleAgreedBy{Начальник 209 ВП МО РФ}{А. И. Бодак}
%\ESKDtitleDesignedBy {Главный технолог}{А. А. Целищев}
\ESKDtitleDesignedBy {Начальник теплотехнического отдела}{А. А. Целищев}
%\ESKDtitleAgreedBy {Нач. отдела надежности}{Д. А. Новиков}
%\ESKDtitleDesignedBy {Ведущий специалист-конструктор-расчетчик}{А. А. Целищев}
%\ESKDtitleDesignedBy {Специалист-конструктор-расчетчик}{В. В. Пшеничный}

\renewcommand{\ESKDtheTitleFieldX}{} % Убрана дата с титульного листа

\renewcommand{\ESKDsectionStyle}{\normalfont\Large\bfseries\MakeUppercase}
\renewcommand{\ESKDsubsectionStyle}{\normalfont\large\bfseries}
\renewcommand{\ESKDsubsubsectionStyle}{\normalfont\normalsize\bfseries}

%\ESKDsectStyle{\section}{\textsc}
%
\newcommand{\No}{\textnumero}
\newcommand{\no}{\textnumero}
\newcommand{\cels}{~$^0C$}
\newcommand{\obmin}{~об/мин}
\newcommand{\mpa}{~МПа}
\newcommand{\kgscm}{~кгс/см$^2$}
\newcommand{\kgsm}{~кгс$\cdot$м}
\newcommand{\kgsmm}{~кгс/мм$^2$}
\newcommand{\kJ}{~кДж}
\newcommand{\J}{~Дж}
\newcommand{\pa}{~Па}
\newcommand{\mm}{~мм}
\newcommand{\kpa}{~кПа}
\newcommand{\ms}{~м/с}
\newcommand{\watt}{~Вт}
\newcommand{\kwatt}{~кВт}
\newcommand{\megawatt}{~MВт}
\newcommand{\qm}{~м$^2$}
\newcommand{\qmm}{~мм$^2$}
\newcommand{\grad}{~$^0C$}
\newcommand{\minut}{~'}
\newcommand{\secund}{~'}

\bibliographystyle{gost2003s}
\setcounter{tocdepth}{2}
%
%\renewcommand{\l@subsection}{\@tocline{2}{0.8cm}{0.8cm}}
\usepackage{soul}
\makeatletter \renewcommand{\@dotsep}{10000} \makeatother % Исключены точки в оглавлении
% \renewcommand{\l@section}{\@dottedtocline{1}{1.5em}{2.3em}}
% \renewcommand{\l@subsection}{\@dottedtocline{2}{1.5em}{2.3em}}
% \renewcommand{\l@subsubsection}{\@dottedtocline{2}{1.5em}{2.3em}}

%\renewcommand{\part}[1]{#1 \refstepcounter{}

%\renewcommand{\thesection}{\thepart.\arabic{section}}

\newcommand{\thepointsection}{\thesection.\arabic{subsection}}
\newcommand{\pointsection}{%
  \par\refstepcounter{subsection}\thepointsection\quad}

\newcommand{\thepointsubsection}{\thesubsection.\arabic{subsubsection}}
\newcommand{\pointsubsection}{%
  \par\refstepcounter{subsubsection}\thepointsubsection\quad}

\newcounter{subsubsubsection}[subsubsection]

\newcommand{\thepointsubsubsection}{\thesubsubsection.\arabic{subsubsubsection}}
\newcommand{\pointsubsubsection}{%
  \par\refstepcounter{subsubsubsection}\thepointsubsubsection\quad}

%создание и автоматическая нумерация списков
\RequirePackage{enumitem}
\renewcommand{\alph}[1]{\asbuk{#1}} % костыль для кирилической нумерации вместо латинской

\setlist{nolistsep} % убираем дополнительные вертикальные отступы вокруг списков
\setenumerate[1]{label=\alph*), fullwidth, itemindent=\parindent,  listparindent=\parindent} 
\setenumerate[2]{label=\arabic*), fullwidth, itemindent=\parindent, listparindent=\parindent, leftmargin=\parindent}

\setitemize{fullwidth, itemindent=\parindent, listparindent=\parindent}

% Настраиваем шрифт разделов в оглавлении
\makeatletter
\renewcommand{\l@section}{\bfseries \@dottedtocline{1}{0em}{1.5em}}
\renewcommand{\l@subsection}{\normalfont \@dottedtocline{2}{0em}{2.5em}}
\makeatother
